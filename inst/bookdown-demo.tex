\documentclass[]{book}
\usepackage{lmodern}
\usepackage{amssymb,amsmath}
\usepackage{ifxetex,ifluatex}
\usepackage{fixltx2e} % provides \textsubscript
\ifnum 0\ifxetex 1\fi\ifluatex 1\fi=0 % if pdftex
  \usepackage[T1]{fontenc}
  \usepackage[utf8]{inputenc}
\else % if luatex or xelatex
  \ifxetex
    \usepackage{mathspec}
  \else
    \usepackage{fontspec}
  \fi
  \defaultfontfeatures{Ligatures=TeX,Scale=MatchLowercase}
\fi
% use upquote if available, for straight quotes in verbatim environments
\IfFileExists{upquote.sty}{\usepackage{upquote}}{}
% use microtype if available
\IfFileExists{microtype.sty}{%
\usepackage{microtype}
\UseMicrotypeSet[protrusion]{basicmath} % disable protrusion for tt fonts
}{}
\usepackage[margin=1in]{geometry}
\usepackage{hyperref}
\hypersetup{unicode=true,
            pdftitle={Archaeological Science with R},
            pdfauthor={Ben Marwick},
            pdfborder={0 0 0},
            breaklinks=true}
\urlstyle{same}  % don't use monospace font for urls
\usepackage{natbib}
\bibliographystyle{apalike}
\usepackage{color}
\usepackage{fancyvrb}
\newcommand{\VerbBar}{|}
\newcommand{\VERB}{\Verb[commandchars=\\\{\}]}
\DefineVerbatimEnvironment{Highlighting}{Verbatim}{commandchars=\\\{\}}
% Add ',fontsize=\small' for more characters per line
\usepackage{framed}
\definecolor{shadecolor}{RGB}{248,248,248}
\newenvironment{Shaded}{\begin{snugshade}}{\end{snugshade}}
\newcommand{\KeywordTok}[1]{\textcolor[rgb]{0.13,0.29,0.53}{\textbf{{#1}}}}
\newcommand{\DataTypeTok}[1]{\textcolor[rgb]{0.13,0.29,0.53}{{#1}}}
\newcommand{\DecValTok}[1]{\textcolor[rgb]{0.00,0.00,0.81}{{#1}}}
\newcommand{\BaseNTok}[1]{\textcolor[rgb]{0.00,0.00,0.81}{{#1}}}
\newcommand{\FloatTok}[1]{\textcolor[rgb]{0.00,0.00,0.81}{{#1}}}
\newcommand{\ConstantTok}[1]{\textcolor[rgb]{0.00,0.00,0.00}{{#1}}}
\newcommand{\CharTok}[1]{\textcolor[rgb]{0.31,0.60,0.02}{{#1}}}
\newcommand{\SpecialCharTok}[1]{\textcolor[rgb]{0.00,0.00,0.00}{{#1}}}
\newcommand{\StringTok}[1]{\textcolor[rgb]{0.31,0.60,0.02}{{#1}}}
\newcommand{\VerbatimStringTok}[1]{\textcolor[rgb]{0.31,0.60,0.02}{{#1}}}
\newcommand{\SpecialStringTok}[1]{\textcolor[rgb]{0.31,0.60,0.02}{{#1}}}
\newcommand{\ImportTok}[1]{{#1}}
\newcommand{\CommentTok}[1]{\textcolor[rgb]{0.56,0.35,0.01}{\textit{{#1}}}}
\newcommand{\DocumentationTok}[1]{\textcolor[rgb]{0.56,0.35,0.01}{\textbf{\textit{{#1}}}}}
\newcommand{\AnnotationTok}[1]{\textcolor[rgb]{0.56,0.35,0.01}{\textbf{\textit{{#1}}}}}
\newcommand{\CommentVarTok}[1]{\textcolor[rgb]{0.56,0.35,0.01}{\textbf{\textit{{#1}}}}}
\newcommand{\OtherTok}[1]{\textcolor[rgb]{0.56,0.35,0.01}{{#1}}}
\newcommand{\FunctionTok}[1]{\textcolor[rgb]{0.00,0.00,0.00}{{#1}}}
\newcommand{\VariableTok}[1]{\textcolor[rgb]{0.00,0.00,0.00}{{#1}}}
\newcommand{\ControlFlowTok}[1]{\textcolor[rgb]{0.13,0.29,0.53}{\textbf{{#1}}}}
\newcommand{\OperatorTok}[1]{\textcolor[rgb]{0.81,0.36,0.00}{\textbf{{#1}}}}
\newcommand{\BuiltInTok}[1]{{#1}}
\newcommand{\ExtensionTok}[1]{{#1}}
\newcommand{\PreprocessorTok}[1]{\textcolor[rgb]{0.56,0.35,0.01}{\textit{{#1}}}}
\newcommand{\AttributeTok}[1]{\textcolor[rgb]{0.77,0.63,0.00}{{#1}}}
\newcommand{\RegionMarkerTok}[1]{{#1}}
\newcommand{\InformationTok}[1]{\textcolor[rgb]{0.56,0.35,0.01}{\textbf{\textit{{#1}}}}}
\newcommand{\WarningTok}[1]{\textcolor[rgb]{0.56,0.35,0.01}{\textbf{\textit{{#1}}}}}
\newcommand{\AlertTok}[1]{\textcolor[rgb]{0.94,0.16,0.16}{{#1}}}
\newcommand{\ErrorTok}[1]{\textcolor[rgb]{0.64,0.00,0.00}{\textbf{{#1}}}}
\newcommand{\NormalTok}[1]{{#1}}
\usepackage{longtable,booktabs}
\usepackage{graphicx,grffile}
\makeatletter
\def\maxwidth{\ifdim\Gin@nat@width>\linewidth\linewidth\else\Gin@nat@width\fi}
\def\maxheight{\ifdim\Gin@nat@height>\textheight\textheight\else\Gin@nat@height\fi}
\makeatother
% Scale images if necessary, so that they will not overflow the page
% margins by default, and it is still possible to overwrite the defaults
% using explicit options in \includegraphics[width, height, ...]{}
\setkeys{Gin}{width=\maxwidth,height=\maxheight,keepaspectratio}
\IfFileExists{parskip.sty}{%
\usepackage{parskip}
}{% else
\setlength{\parindent}{0pt}
\setlength{\parskip}{6pt plus 2pt minus 1pt}
}
\setlength{\emergencystretch}{3em}  % prevent overfull lines
\providecommand{\tightlist}{%
  \setlength{\itemsep}{0pt}\setlength{\parskip}{0pt}}
\setcounter{secnumdepth}{5}
% Redefines (sub)paragraphs to behave more like sections
\ifx\paragraph\undefined\else
\let\oldparagraph\paragraph
\renewcommand{\paragraph}[1]{\oldparagraph{#1}\mbox{}}
\fi
\ifx\subparagraph\undefined\else
\let\oldsubparagraph\subparagraph
\renewcommand{\subparagraph}[1]{\oldsubparagraph{#1}\mbox{}}
\fi

%%% Use protect on footnotes to avoid problems with footnotes in titles
\let\rmarkdownfootnote\footnote%
\def\footnote{\protect\rmarkdownfootnote}

%%% Change title format to be more compact
\usepackage{titling}

% Create subtitle command for use in maketitle
\newcommand{\subtitle}[1]{
  \posttitle{
    \begin{center}\large#1\end{center}
    }
}

\setlength{\droptitle}{-2em}
  \title{Archaeological Science with R}
  \pretitle{\vspace{\droptitle}\centering\huge}
  \posttitle{\par}
  \author{Ben Marwick}
  \preauthor{\centering\large\emph}
  \postauthor{\par}
  \predate{\centering\large\emph}
  \postdate{\par}
  \date{2016-10-14}

\usepackage{booktabs}

\begin{document}
\maketitle

{
\setcounter{tocdepth}{1}
\tableofcontents
}
\chapter{Archaeological Science with
R}\label{archaeological-science-with-r}

This book will teach you how to get started doing archaeological science
with R: You'll learn how to get your data into R, get it into the most
useful structure, transform it, visualise it and model it. In this book,
you will find a practicum of skills for archaeological science.

To be published by XXX in XXX 201X.

\subsection{Notes to self}\label{notes-to-self}

\begin{itemize}
\tightlist
\item
  To render the book to PDF and HTML (since it's
  \href{https://github.com/rstudio/bookdown/issues/215}{nested in a
  project directory}), setwd and then
  \texttt{source(\textquotesingle{}\_render.R\textquotesingle{})}\\
\item
  To get a
  \href{https://bookdown.org/yihui/bookdown/serve-the-book.html}{live
  preview of the book in HTML}:
  \texttt{bookdown::serve\_book(dir\ =\ ".",\ output\_dir\ =\ "\_book",\ preview\ =\ TRUE,\ in\_session\ =\ TRUE,\ daemon\ =\ FALSE)}
\end{itemize}

\part{Motivation}\label{part-motivation}


\chapter{Introduction}\label{introduction}

\section{Overview}\label{overview}

The goal of ``Archaeological Science with R'' is to give you a solid
foundation into using R to do archaeological science. By archaeological
science, I mean, systematic, objective, and empirical research into past
human behaviours using material remains and traces. The goal is not to
be exhaustive, but to instead focus on what I think are the critical
skills for doing archaeological science with R. Some archaeologists
already use R in an \emph{ad hoc} way, for a quick plot here, or a
linear model there. But this book will show you how R can be at the
center of your entire research workflow, from when you start working
with raw data until your final thesis, report, or manuscript is
complete.

\section{Why do archaeologists need to learn to
code?}\label{why-do-archaeologists-need-to-learn-to-code}

This book aims to solve a specific problem. The problem is that the
majority of archaeologists receive little or no training in scientific
programming for data analysis and visualization, and yet they routinely
analyse and visualise data. A command-line interface program such as R
is ideal for this kind of work, and yet is unfamiliar and exotic to most
archaeologists. A command-line interface refers to interacting with
software by sending instructions to the program as lines of plain text.
Instead, most archaeologists use Microsoft Excel, SPSS, or similar
commercial point-and-click software. There are four problems with this.

\begin{enumerate}
\def\labelenumi{\arabic{enumi}.}
\item
  It usually results in a lot of time-consuming repetition, like
  copying-and-pasting between sheets in Excel, and between Excel and
  Microsoft Word. In recent years, biologists have seen great increases
  in volumes of genomic data, due to improvements in sequencing
  technology. This has lead them to search for more efficient ways to
  analyse their data, and automate their analyses. They found that
  spreadsheet programs did not provide enough flexibility to conduct
  their analyses. As a result, many have turned to R, Python, and
  similar programming languages to overcome the limitations and
  inefficiencies of Excel. The key detail here is a shift from the
  point-and-click interface of a spreadsheet program, to the
  command-line interface of a programming language where we supply
  instructions to the computer in plain text.
\item
  It limits the development of new methods because the you are limited
  to what is available in the commercial software. With most commercial
  software packages, you are limited to the suite of functions they
  choose to make available to you. With an open source programming
  language, you and anyone else are free to implement new methods. R has
  extensive functions for data analysis. This includes features likes
  missing values and subsetting. But more importantly, R has a large set
  of packages (currently \textgreater{}6000) for quantitative analysis,
  visualisation, and importing and manipulating data. For most
  archaeologists, whatever analysis or plot you are attempting, chances
  are that someone has already tried it, or something very similar, and
  made the code available in an R package. R is also the lingua franca
  for researchers in statistics, who will often publish an R package to
  accompany their scholarly articles. This often means immediate access
  to the very latest statistical techniques and implementations.
\item
  Commercial software also makes it difficult to demonstrate and ensure
  reproducibility due to the ephemeral nature of point-and-click
  interfaces. Point-and-click interfaces are familiar and easy to use
  for most people because they are very common in software programs. But
  it is very challenging to efficiently record a sequence of complex
  clicks so that another person (or you in the future) can unambiguously
  repeat the procedure. This means it is difficult to make your analyses
  reproducible if all your work is done with a mouse. It is not
  practical to completely abandon using a mouse when using a computer,
  but by using a command-line program such as R, we can record the most
  important steps of the analysis workflow in plain text code. Then we
  have an detailed record of the steps in our analysis that we can
  easily share with others, and re-use ourselves, months or years into
  the future
\item
  Commercial software limits transparency in research because the
  algorithms behind the functions are not available for convenient
  inspection and modification. Staticians have long noted that Excel has
  flawed statistical procedures
  \citep[\citet{yalta2008accuracy}]{mccullough2008accuracy}, and
  introductory texts on statistics warn students not to use Excel when
  the results matter \citep[eg.][]{keller2000applied}. With an open
  source program such as R, you can easily inspect and alter the
  algorithms behind every operation. R in particular has the added
  advantage of being one of the most accurate software programs for
  statistical analysis
  \citep[\citet{keeling2007comparative}]{almiron2009}.
\end{enumerate}

Using a command-line program such as R goes a long way towards solving
these four problems.

This book is complimentary to the major textbooks and handbooks of
quantitative archaeology (Baxter 2003; 1994, Van Pool and Leonard 2011,
Drennan 2009, Shennan 1997). These are excellent for relevant
statistical theory, discussion, and examples, and this book does not
replace them on your shelf. However, those books have three limitations
that we will address in the following chapters.

\begin{enumerate}
\def\labelenumi{\arabic{enumi}.}
\item
  First is the absence of `plug-and-play' examples, leading the reader
  to invest substantial effort to implement a method described using
  formulae, creating many opportunities for error. Van Pool and Leonard
  2011 instruct their reader to do their statistical analyses by hand,
  an impractical and outmoded recommendation when computers are readily
  available. The amount of effort required by these texts is prohibitive
  to rapid exploration and experimentation of data using new methods.
  This book includes reproducible examples using real data sets so you
  can easily step-through the analysis to explore the effects of
  different parameters, and easily interchange your own data.
\item
  The existing books give little coverage to intuitive and
  methodologically more robust methods such as resampling and Bayesian
  analyses, favoring traditional parametric methods. This reflects a
  time when computational power was scarce, and computations often done
  by hand or with a calculator (cf Fletcher and Lock 1991). Modern
  computers can now easily handle resampling and Bayesian methods, and R
  is unique in having a mature set of methods for computing common tests
  in these frameworks. These methods are also increasingly common in the
  published research literature. This book introduces these alternative
  approaches to give you more options in your analytical toolkit.
\item
  The currently available books are silent on the practical mechanics of
  many common data analysis and visualization tasks in archaeology that
  are not traditionally considered statistics. This includes displaying
  and analyzing stratigraphic information from excavation and spatial
  data from surveys. Learning a tool-chain for these tasks is
  traditionally done on your own, and often at substantial expense with
  proprietary software. This book demonstrates how to write simple
  programs that are especially useful for archaeologists. By doing this,
  and providing code for other common tasks, this book addresses the
  need for instruction in a comprehensive open source tool-chain for
  these common tasks in archaeological science. It does this by
  presenting a reproducible research environment to show how R and its
  contributed packages can be used for start-to-finish research into
  survey, excavation, and laboratory data.
\end{enumerate}

My hope is that a practical and accessible introduction to reproducible
research, such as this book aims to provide, will improve openness and
transparency in archaeology generally. It will contribute towards
creating a community of researchers where it it normal and routine to
publish code and data (after appropriate precautions are taken to
protect sensitive information) with reports and publications, so that we
can engage more deeply with, and learn more efficiently from, each
other.

This is not a book of detailed discussions of statistical theory, and
you will be referred to other texts on technical details of algorithms,
etc. This is a book introducing R for practical and efficient
implementations of common tasks in archaeological science, and serves as
a springboard to more advanced R programming. The specific topics
covered in this book are:

\begin{itemize}
\item
  Implementing reproducible research with literate programming as a
  practice that is good for science generally, and good for your
  individual productivity
\item
  Working with common archaeological data structures, ingesting them
  into R and manipulating them from messy formats to tidy formats ready
  for analysis
\item
  Visualising spatial data by making maps, doing spatial analysis and
  site classification
\item
  Visualising and analysing stratigraphic data from archaeological
  excavation
\item
  Working with relative and absolute chronologies by doing seriation,
  and calibrating, analysing and visualising radiocarbon dates
\item
  Working with data from stone artefacts, faunal remains, pottery,
  glass, and metal artefacts
\item
  Compositional analysis using cluster and principle components analyses
  of multivariate data
\item
  Simplifying collaborative research with version control
\end{itemize}

\section{Why R?}\label{why-r}

If you are new to R, you might wonder what the appeal is, especially
when there are so many programming languages in common use. Some of the
characteristics that drew me to learn R include:

\begin{itemize}
\item
  It's free, open source, and available on every major type of computer.
  As a result, if you do your analysis in R, anyone can easily reproduce
  it.
\item
  It is specialised for use with statistical analysis and data
  visualisation. R was originaly develped by statisticians, and
  continues to be widely used by professional and academic
  statisticians. This means that there are strong links between the
  statistical literature and R code. Many of its algorithms have been
  vetted by publication and thoroughly tested through extensive use. It
  is easy to go from statistical theory to practice using R.
\item
  An active, supportive and generally progressive community. It is easy
  to get help from experts on the R-help mailing list, Stack Overflow,
  or subject-specific mailing lists. You can also connect with other R
  learners via social media, and through many local user groups. I have
  found Stack Overflow and twitter to be particularly useful sources of
  information. Package authors are often pleased to see their packages
  used by others, and willing to answer questions about your use of
  their package.
\item
  Flexible tools for communicating your results. R packages make it easy
  to produce html, pdf, or Microsoft Word documents that include text,
  tables and figures, all generated from R code. I find this to be a
  huge time-saver, especially when weeks or months pass between working
  on a project, and I cannot remember all the details of the last work I
  did. When I look over my R code I can quickly recall my analysis plan
  and resume from where I stopped.
\end{itemize}

\section{Who should read this book}\label{who-should-read-this-book}

This book is written for archaeologists who are keen to expand their
analytical horizons and gain access to new methods and new, more
efficient, ways of organising their research process. No prior knowledge
about R, computer science or programming is expected, but a curiosity
about statistics would be an advantage, as well as a readiness to read
beyond this book to make decisions about the suitability of statistical
methods for your specific research questions.

\section{Prerequisites}\label{prerequisites}

To run the code in this book, you will need to have R installed on your
computer, as well as RStudio, an application that makes it substantially
easier to use R. Although I have been interested in R for many years, it
was only after RStudio was first released that R replaced Excel and SPSS
for me. Both R and RStudio are free to download and easy to install. You
can install it on most recent versions of Windows, OSX (Mac) or Linux.
When used together, R and RStudio will help you to have a smooth and
efficient experience when programming. RStudio provides many features
that greatly simplify programming with R (my favourite is
tab-completion). I do not recommend using R without RStudio, especially
for the beginner.

You should download and install both R and RStudio before attempting to
follow any of the examples in this book. If you have installed these
programs on your computer in the distant past, you should update them
now (new versions of R are released every six months).

\subsection{R}\label{r}

To install R, visit
\href{http://cran.r-project.org}{cran.r-project.org}. CRAN is the
`Comprehensive R Archive Network' which hosts the official releases of
the R software and documentation. Then click the link that matches your
operating system. What you do next will depend on your operating system.
You will need to install a development environment, this is also
specific to your operating system. The development environment provides
some additional programs that give you access to more advanced features
of the R language.

Here's a summary of the basic steps to download and install R for the
most common operating systems:

\begin{itemize}
\item
  Mac users should click the most current release. This will be the
  \texttt{.pkg} file at the top of the page. Once the file is
  downloaded, double click it to open an R installer. Follow the
  directions in the installer to install R. You should also install
  Xcode from the Mac App Store to install the development environment.
\item
  Windows users should click ``base'' and then download the most current
  version of R, which will be linked at the top of the page. You should
  also install
  \href{http://cran.r-project.org/bin/windows/Rtools/}{Rtools} to
  install the development environment.
\item
  Linux users will need to select their distribution and then follow the
  distribution specific instructions to install R.
  \href{https://cran.r-project.org/bin/linux/}{cran.r-project.org}
  includes these instructions along side of the files to download.
\end{itemize}

If you are not sure about these download and install steps then you
should research the general methods of installing software for your
operating system, which are beyond the scope of this book.

After installing R on your computer, you may be tempted to open it and
start exploring. I don't recommend this because the R user interface is
not intuitive for novices. Instead, read on to continue with downloading
and installing RStudio. Then start your explorations in RStudio, which
has a more user-friendly interface.

\subsection{RStudio}\label{rstudio}

Once you have R installed, it is time to download RStudio. To download
RStudio, visit
\href{http://www.rstudio.com/download}{www.rstudio.com/download}.

Choose the installer that matches your operating system. Then click the
link to download the application. Once you have the application,
installation is the same as for most programs on your computer. Once
RStudio is installed, open it as you would open any other application.
It should look something like this:

{[}insert RStudio\_first\_view about here{]}

When you start RStudio for the first time, you'll see your screen
divided into three or four window panes. If you don't see four window
panes similar to the image above, go to File -\textgreater{} New File
-\textgreater{} R Script to show all four panes. Here is an explanation
of what you see in the figure above (for more details about the RStudio
interface, see the RStudio website
\url{https://www.rstudio.com/products/rstudio}, Verzani
\citeyearpar{verzani2011getting}, Gandrud \citeyearpar{Gandrud2015}, and
Racine \citeyearpar{racine2012rstudio}):

\begin{itemize}
\tightlist
\item
  The top left pane is the text editor, known as the `source'. This is
  where you write and edit code and text. The text editor in RStudio is
  analogus to a word processor such as Microsoft Word, but has many
  additional convienent features for writing and editing code. The text
  editor does not include any tools for formatting your text \emph{in
  situ}, for example you cannot make words appear in the editor with
  bold or italic characters. This is a deliberate design, and is typical
  among text editing programs used for writing code. When we use
  Microsoft Word, we are using a ``what you see is what you get'' editor
  -- we edit a document in a form resembling the final printed or
  presented document. The RStudio text editor is typical of text editors
  designed for coding in taking a different approach to formatting
  documents. When working with a plain text editor, we type unformatted
  plain text and code, and among the text we include instructions on how
  the final document should appear. We will discuss this further in
  Chapter XX, but as a quick example, if we wanted a word in bold, we
  would code it like \texttt{**this**}. In this example, the two asterix
  symbols will be interpreted by our computer when we convert our source
  document into a final version, such as a PDF, to produce a document
  with \textbf{this} word in bold.
\end{itemize}

This method of creating documents is very common in disciplines such as
mathematics and physics, where complex equations are written using code
because for many people it is easier to write code to produce equations
than use a mouse to click on menus and buttons to select the special
characters needed for equations. LaTeX is widely used code-based
document preparation system, with particular strengths in complex
formatting needed by technical scientific documents. Although highly
flexible, LaTeX is also complex and verbose, requiring a lot of typing
to add the formatting code to your document. But don't worry, in this
book we will use a much simpler document preparation system, known as
markdown. We will discuss markdown in more detail in chapter XX, for now
the main thing to know is that we use the text editor to write code and
text.

\begin{itemize}
\item
  The top right pane can have several tabs, depending on what you are
  doing. Most of the time you will see an `Environment' tab and a
  `History' tab. The Environment tab lists the objects that are
  currently available in your R session. For exampe, if you have
  executed code that reads an Excel spreadsheet into R and stores it as
  an R object, then you will see this object listed in the Environment
  tab. This tab is useful because you can get basic information at a
  glance about the size of objects that you make (e.g.~how many rows and
  columns). This is convienent for quickly checking the output of your
  computations so you can see if your code is working as expected. The
  History tab shows the code you have previously executed in your R
  session. I rarely use this tab because I write code into the text
  editor and save the code to my computer, so I always have a copy of
  the code in the text editor. But if you are experimenting with some
  code in an expedient fashion, you can use the History tab to browse
  your previous commands to reuse and edit. You can also save the
  history of your R commands to a text file, but I don't recommend this
  as it is not an efficient method for keeping track of your work. A
  better strategy is to write code in the text editor, interspersed with
  text that explains what the code is doing, and save that document to
  your computer. We'll discuss this strategy more in chapter XX.
\item
  The lower right pane is another multi-tab pane where plots appear, the
  most important are the Plots tab and the Files tab. The Plots tab
  allows you to conveniently browse the plots that you produce. You can
  navigate back to previous plots created in your current session, and
  interact with plots by zooming and, in some cases selecting, rotating,
  etc. The Plots tab has an `Export' button to easily save your plots as
  image files on your computer (in a variety of formats), or copy to
  your clipboard for a quick copy-paste transfer. These are useful
  functions for exploring pictures of your data and iterating towards a
  publication-quality plot.
\item
  The lower left pane is the console where code is executed. There are
  no buttons specific to the console, so direct interaction with this
  pane is limited. The most important part of this pane is the prompt,
  which looks like this \texttt{\textgreater{}}, and when this pane is
  active, there is a blinking cursor at the prompt. The space where the
  cursor is in is called the \texttt{command\ line}, where you type,
  paste, or send code. When you type at the prompt and press Enter, you
  are submitting code to be interpreted by R, and R will usually return
  the result in the console. Here is one of the simplest possible
  examples of using the console:
\end{itemize}

\begin{verbatim}
> 1 + 1
[1] 2
\end{verbatim}

In this example, I placed my cursor at the prompt, typed
\texttt{1\ +\ 1} with my keyboard, and then pressed Enter. The R
interpretor returned the result, 2, directly below my input. The
\texttt{{[}1{]}} indicates that this is the first item returned by the
interpretor. At first this might seem unessecary, but this indexing is
helpful when working with more complex calculations. Most of the time
you wont type directly at the prompt because it is not efficient for
more complex code. Instead, you'll type code in the text editor, and
send that code from the text editor to the prompt by pressing they
Control + Enter keys simutanteously (on OSX: Command + Enter).

There are two useful features of the console that will save you time
when working with R. First is the up arrow key, which allows you to
re-run previous commands. If you place your cursor at the prompt, and
press the up arrow key, you can browse the previously executed lines of
code. Once you find the code you want to re-run, you can edit it in the
console if you wish, and press Enter to send it to the interpreter.
Second is the small arrow icon at the right side of the title bar of the
console. If you look at the top of the console pane, you'll see the word
``Console'', then you'll see a path to a folder on your computer, then
at the end of that, you'll see a little arrow icon pointing right. The
path on your computer in this console title bar is your `Working
Directory'. If you click on the small arrow, it will switch the Files
pane to show the contents of your Working Directory.

The Working Directory is an important concept to understand because it
is unlike anything you might have seen in Excel or SPSS. This is the
directory where R is currently working, for example, where R will look
when you read files into R, or save output from R to your computer. By
default, your R working directory was probably set to an inconvienent
directly when you installed the program, but this is easily changed. I
find that the simplest method is to go to the RStudio toolbar menu,
click on `Session', then `Set Working Directory' then `To Source File
Location'. You may see in other people using R code to set the working
directory, but I don't recommend this. If you use code to set the
working directory, your code will only work on your computer in its
current configuration, and so your code is not portable and not robust
to change. If you reorganise your files, or give your code to a
collaborator, the working directory will no longer match what is written
in the code. This can lead to errors and frustration. So while the
concept of the Working Directory is important to know about for working
with R, you should not include it in your R code.

The location and combination of panes and tabs in RStudio are
customisable via the toolbar menu (Tools -\textgreater{} Global Options
-\textgreater{} Pane Layout). For example, I prefer to have to console
on the upper right, and I often minimize some of the panes so I only see
the text editor and the console.

\section{Key terms to understand (don't skip this
bit!)}\label{key-terms-to-understand-dont-skip-this-bit}

Now we've established the basic motivation for learning R to do
archaeology, and introduced the layout of the software, we should
introduce some key terms and conventions that appear frequently in this
book. Becoming familiar with these terms will help you with the basic
tasks of finding your way around when starting to use R.

We have already discussed the prompt and command line above, and we've
used the word \texttt{code}, but without really defining it.
\texttt{Code} refers to anything written in a programming language, from
a single function to thousands of lines. Code often contains
\texttt{comments}, which you can recognise in the R language by a
\texttt{\#} symbol at the start of the line. Code comments are to help
you and other people understand the purpose of the code. The \texttt{\#}
symbol tells the R interpretor to ignore to everything to the right of
that symbol, and skip to the next line of code. A \texttt{script} is a
file that contains code, and usually R script file names ends with
\texttt{.R} or \texttt{.r}. You can open these files in any program that
edits text, for example Notepad on Windows (or Notepad++) or TextEdit on
OSX (or Sublime Text).

A \texttt{function} refers to a group of code that carries out
instructions to do useful work. For example, the function
\texttt{mean()} computes the mean of a set of numbers. In this book you
will always be able to recognise a function because they will always be
followed by a pair of parentheses. A function is a handy way to organise
code, because a single function can cause hundreds of lines of code to
run to produce a result. Rather than running those hundreds of lines,
one-by-one, over and over, we can simply type the function names that
invokes them, and save ourself a lot of typing. Usually we want to save
the output of a function so we can use in other functions, or for
plotting. For example, we might want to compare the mean of the lengths
of two assemblages of artefacts. To keep the results of function, we
\texttt{assign} the output to an \texttt{object}. Here, we assign the
output of the \texttt{mean()} function to an object that we arbitrarily
call \texttt{x}, like so \texttt{x\ \textless{}-\ mean(c(4,\ 7,\ 12))}.
This short piece of code computes the mean of 4, 7, and 12 (ie. (4 + 7 +
12)/3), and then assigns, or stores, the result in a object called
\texttt{x} (we do not need to create \texttt{x} in advance). The
\texttt{c} is a function to combine numbers into a set called a
\texttt{vector}, which is a fundamental data object type in R. We can
then use \texttt{x} later in our workflow for other tasks, rather than
recomputing the mean repeatedly.

There's lot more vocabularly to come, but these are the key terms that
you need to know to make sense of the rest of this book.

\subsection{Packages}\label{packages}

Functions are at the heart of working with R, and are part of the reason
for R's great versatility. They save a lot of time by minimizing
copy-and-pasting and typing, so it's worth to invest some effort into
learning how to write your own functions. Functions are typically
organised into \texttt{packages}, which you can download to extend the
usefulness of R. We will use packages extensively in this book, and
briefly discuss how and why you might write your own packages. There are
two common methods to install packages for R. For either method you need
an internet connection, an R session (by opening RStudio), and to run a
command at the command line.

The most common way to install R packages is to run the command,
\texttt{install.packages()} at the R prompt. This will download the
package from the repository at
\href{http://cran.r-project.org}{cran.r-project.org} and install them in
your system library.

Try this yourself by opening R and running these commands:

\begin{Shaded}
\begin{Highlighting}[]
\CommentTok{# install a single package}
\KeywordTok{install.packages}\NormalTok{(}\StringTok{"devtools"}\NormalTok{)}
\CommentTok{# install multiple packages at one time}
\KeywordTok{install.packages}\NormalTok{(}\KeywordTok{c}\NormalTok{(}\StringTok{"dplyr"}\NormalTok{, }\StringTok{"ggplot2"}\NormalTok{, }
                   \StringTok{"knitr"}\NormalTok{, }\StringTok{"readr"}\NormalTok{, }\StringTok{"rmarkdown"}\NormalTok{, }
                   \StringTok{"scales"}\NormalTok{, }\StringTok{"stringr"}\NormalTok{, }\StringTok{"tidyr"}\NormalTok{))}
\end{Highlighting}
\end{Shaded}

Most of the packages we will use in this book are stored on
\href{http://cran.r-project.org}{cran.r-project.org}, but are other
online repositories for R packages. For example, many R package authors
use \href{http://www.github.com}{www.github.com} to make their packages
publicly available.

You can install packages stored on github with the
\texttt{install\_github()} function in the \texttt{devtools} package
(which we installed using the commands above). To use the
\texttt{install\_github()} function, pass it a character string with the
form
\texttt{"\textless{}github\ username\textgreater{}/\textless{}github\ repository\ name\textgreater{}"}.
For example, we can install the package \texttt{ggbiplot} to plot the
output of a Principle Components Analysis like this:

\begin{Shaded}
\begin{Highlighting}[]
\NormalTok{devtools::}\KeywordTok{install_github}\NormalTok{(}\StringTok{"vqv/ggbiplot"}\NormalTok{)}
\end{Highlighting}
\end{Shaded}

When R installs a package, it downloads the package to your system
\texttt{library}. You must use the \texttt{library()} function to make
the contents of the package available to your current R session. It's
good practice to have the \texttt{library()} functions among the first
few lines of your code, so other users can quickly see what packages
they will need to have to prepare to run your code. For example, to use
the functions in the \texttt{dplyr} package, you would need to first run

\begin{Shaded}
\begin{Highlighting}[]
\KeywordTok{library}\NormalTok{(}\StringTok{"dplyr"}\NormalTok{)}
\end{Highlighting}
\end{Shaded}

You will need to rerun this \texttt{library()} command each time you
open a new R session that uses functions from the \texttt{dplyr}
package.

\section{Getting help}\label{getting-help}

Developing fluency in a language like German or Chinese takes time and
has ups and downs. Learning a programming language like R is a similar
process, and you should anticipate ups and downs as R becomes an
increasingly central part of your workflow. Fortunately, there are a
number of options for seeking help when you get stuck, or have a
question. The simplest case is when you execute a line of code, only to
get a cryptic error message in the console. A good first reponse is to
copy the text of the error message to your clipboard, and paste it into
a Google search box in your web browser. In most cases, the top search
results will be from \href{http://stackoverflow.com/}{Stack Overflow} or
one of the \href{https://www.r-project.org/mail.html}{R mailing lists}.
Sometimes this will get you a quick solution. However, sometimes error
messages are too generic, or too specific, and you can browse through
many pages of search results without getting any useful ideas to solve
your problem. When that happens, there are a few places online where you
can give some more detail about your problem, and get more relevant
responses.

Stack Overflow is a free question-and-answer website that is focused on
computer programming. Participants receive points for asking clear
questions, and for giving useful answers. This adds a competitive
element to participating in the Q\&A, and often when you ask a
well-formed question, several answers will appear very quickly from
participants eager to earn points. The R mailing lists are more
traditional email-based fora, with archives at several places online.
Highly skilled R programmers and R-using scientists from a variety of
disciplines are active on both the R mailing lists and Stack Overflow. I
prefer Stack Overflow because responses to my questions typically come
quicker, and I find it easier to see when someone else's question has
been answered, compared to browsing the email list archives.

I find that in the course of writing my question to submit to Stack
Overflow, the process of simplifying my code into a small,
self-contained example leads me to discover the cause of my problem
(very often it's a misplaced comma or bracket). Then I can answer my
question before I need help from others. Similarly, I often find that
many of my questions have already been asked and answered on Stack
Overflow. The challenge in this case is to recognise how similar an
existing question is to your current problem, and thus how useful the
answers are to your specific issue.

If you are sure that your problem is new and unique, and you want to
submit a question, here is some general advice to help you get a useful
answer quickly from the Stack Overflow and R mailing lists communities:

\begin{itemize}
\item
  Spend some time browsing previous posts to understand the community
  norms. Online communities have specific cultural values that are often
  strongly held. While some of these are spelled out in posting guides,
  many are also unwritten, and can only be learnt through mindful
  reading of previous correspondence. If you are familiar with the
  norms, you are more likely to have an efficient and satisfying
  interaction with the community.
\item
  Make sure you have the latest version of R and of the package (or
  packages) you are having problems with. It may be that your problem is
  the result of a recently fixed bug.
\item
  Spend some time creating a reproducible example. This mean a
  simplified version of your problem that another person can copy and
  paste from their browser into their R console to reproduce the error
  that you see. This can be quite an art-form, but the basic ingredients
  for a good reproducible example are:
\end{itemize}

\begin{enumerate}
\def\labelenumi{\arabic{enumi}.}
\tightlist
\item
  The packages you are using\\
\item
  A small data set. R comes with many example data sets built-in, if you
  run \texttt{data()} you can see a list, and these are very convenient
  to use for reproducible examples. If you want to include your own
  data, you can use \texttt{dput()} to generate R code that another
  person can copy and paste to recreate your data set.\\
\item
  Code. Your code should be easy to read, with spaces between operators
  (+, -, *, etc.) and after commas, and conformant to a style guide,
  such as Hadley Wickham's \citeyearpar{wickham2014advanced}
  \href{http://adv-r.had.co.nz/Style.html}{style guide in Advanced R}.
  Your code should includes lines of commentary, which begin with
  \texttt{\#}, to help others understand your problem.\\
\item
  A description of your R environment. The usual way to communicate your
  R environment (ie. your operating system type and version), is to
  include the output from \texttt{sessionInfo()} into your question.
\end{enumerate}

Don't worry if some of these terms are unfamiliar at the moment. The
main thing to know at this point is that help is available, and that the
quality of help you receive is proportional to the effort you spend
seeking it.

\subsection{Draft TOC}\label{draft-toc}

\begin{enumerate}
\def\labelenumi{\arabic{enumi}.}
\tightlist
\item
  Introduction
\end{enumerate}

\begin{itemize}
\tightlist
\item
  R
\item
  RStudio
\end{itemize}

\begin{enumerate}
\def\labelenumi{\arabic{enumi}.}
\setcounter{enumi}{1}
\tightlist
\item
  Reproducible research \& Literate programming
\end{enumerate}

\begin{itemize}
\tightlist
\item
  markdown
\item
  R projects
\end{itemize}

\begin{enumerate}
\def\labelenumi{\arabic{enumi}.}
\setcounter{enumi}{2}
\item
\item
\end{enumerate}

\chapter{Implementing reproducible
research}\label{implementing-reproducible-research}

\section{Overview}\label{overview-1}

Four factors have lead to growing concerns about the reproducibility of
scientific results in modern science. The first factor is reruns of
landmark biomedical experiments that have failed to produce the same
results as the initial publication of the experiment
\citep{Prinz2011drugtargets, Begley2012clinical}. The second factor is
the discovery of high-profile studies where the published results were
flawed due to simple errors data
analysis\citep{baggerly2009deriving, Herndon2013austerity}. Third, there
has been a sharp rise in the number of scholarly publications that have
been retracted due to discoveries of misconduct such as fraudulent data
and plagiarism \citep{Vannoorden2011retractions, Steen2013retractions}.
Finally, there has been a growth of data-intensive research that use
massive computer simulations or collect enourmous amounts of data from
vast numbers of sensors (ie. astronomy and oceanography). These events
have stimulated extensive discussion in many disciplines about how to
improve the reproducibility of scientific research. Some of this
discussion has resulted in calls from diverse dsicplines
\citep{nosek2015open} for publications to be accompanied by both the raw
data that generated the figures in the paper
\citep{reichman2011challenges}, and the programming code that shows the
steps taken in the data analysis
\citep{barnes2010publish, Ince1012opencomputer, Morin2012blackbox}.
Related to these calls are a manifesto-like journal articles by groups
of researchers that call for substantial changes in the research
workflow to improve reproducibilty and transparency
\citep{wilson2014best, Hampton2015tao, Sandve2013tenrules}.

In this chapter I show how archaeologists can respond to these calls for
increased reproducibilty and transparency. I sketch out how R can be
used to enable reproducible research through literate programming.
Literate programming refers to the use of a computing environment for
authoring mixed natural (eg. English) and computer (eg. R) language
documents \citep{Schulte2012literate}. Conventiently, R and RStudio are
well-equipped to support literate programming, so the practice is
relatively easy to learn. The second section of this chapter introduces
version control as a principle and tool for enhancing transparency and
collaboration. Version control of literate programming files allows you
to keep track of all the changes that you and your co-authors make to
your files. Using the Git software for version control, you and your
collaborators can safely create versions of your files to experiment
with, and either abandon the results, merge them with the main versions,
or revert to an earlier version.

\section{What does it mean to do reproducible
research?}\label{what-does-it-mean-to-do-reproducible-research}

Before showing how we can make our research more reproucible, we should
explore exactly what we mean by `reproducible'. There is some confusion
surrounding recent use of the term `reproducibility' in biology and
computer science. The confusion results from using the term at two quite
different parts of the research pipeline.

The first part is empirical reproducibility, also called
`replicability'. For an archaeologist, this means evaluating a
previously obtained result by re-excavating a site that another
researcher had previously excavated, studying a museum collection that
had previously been analysed by another person, or repeating an
experiement that has been previously published. Empirical
reproducibility is about returning to the source of the data and
physically collecting the data again. Archaeologists do this routinely,
for example, we return to famous sites such as Star Carr, Catal Huyuk,
and Niah Cave to excavate them again. Usually this happens decades after
the original excavation that make the site prominent. This is not exact
empirical reproducibility, because we do not use the exact procedures of
the previous workers, but instead use modern field methods to extract
more data from the site, and improve upon and extend the earlier work.
But the general idea of testing the previous claims about the site is
usually the key motivation for returning to it.

The second use of `reproducibility' is computational and statistical
reproducibility. This refers to redoing the calculations of another
researcher's quantitative results using their original datasets and
methods. This type of reproducibility does not involve independent data
collection, but instead uses the methods and data collected by the
original investigator. This kind of reproducibilty is not something that
is familiar to most archaeologists. Our publications rarely include
enough of the data to enable reanalysis, and our methods are rarely
presented in enough detail to allow another person to independantly
repeat our analysis unabimguously. This type of reproducibility is
becoming an important issue for recent archaeological research due to
advances in technology and the rapid spread of computational methods for
many kinds of basic archaeological analysis. In my view, the short
answer to improving computational and statistical reproducibility in
archaeology is for archaeologists to include data files and programming
code with their publications. These extra documents will allow another
researcher to see all the steps from when the raw data entered the
researcher's computer to the end point where these data were summarised
and visualised in a journal article.

In this chapter I describe two principles, literate programming and
version control. These are dervied from computer science research and
have been adopted by other disciplines, such as ecology and political
science, to enable computational and statistical reproducibility. I also
demonstrate the current software tools (Rmarkdown, knitr, and Git) that
allow us to put these principles into practice. Over time, these
software tools will change, and perhaps be replaced by something
completely different. However, the principles that they are based on
will endure, and are worth becoming familiar with as key concepts in
computational and statistical reproducibility.

\url{http://languagelog.ldc.upenn.edu/nll/?p=21956}

\section{Literate programming}\label{literate-programming}

\section{Version control}\label{version-control}

\bibliography{packages.bib,book.bib}


\end{document}
