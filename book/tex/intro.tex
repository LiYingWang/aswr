\chapter{Welcome}

\section{Overview}

The goal of ``Archaeological Science with R'' is to give you a solid
foundation into using R to do archaeological science. By archaeological
science, I mean, systematic, objective, and empirical research into past
human behaviours using material remains and traces. The goal is not to
be exhaustive, but to instead focus on what I think are the critical
skills for doing archaeological science with R. Some archaeologists
already use R in an \emph{ad hoc} way, for a quick plot here, or a
linear model there. But this book will show you how R can be at the
center of your entire research workflow, from when you start working
with raw data until your final report is complete.

The specific problem is that this book aims to solve is that the
majority of archaeologists receive little or no training in scientific
programming for data analysis and visualization. A command-line
interface program such as R is unfamiliar and exotic to most
archaeologists, yet R is well-equipped to perform almost every
analytical and visualization task an archaeologist could need. Instead,
most archaeologists use Microsoft Excel, SPSS, or similar commercial
point-and-click software. The problem with this is that is it limits the
development of new methods because the you are limited to the built-in
suite. Commercial software also makes it difficult to demonstrate and
ensure reproducibility due to the ephermeral nature of point-and-click
interfaces. Commercial software limit transparency in research because
the algorithms behind the functions are not available for inspection.

The major textbooks and handbooks of quantitative archaeology (Baxter
2003; 1994, Van Pool and Leonard 2011, Drennan 2009, Shennan 1997) are
excellent for relevant statistical theory, discussion, and examples, but
have three limitations that the proposed book will address.

First is the absence of `plug-and-play' examples, leading the reader to
invest substantial effort to implement a method described using
formulae, creating many opportunities for error. Van Pool and Leonard
2011 instruct the reader to do their statistical analyses by hand, an
impractical and outmoded recommendation. The amount of effort required
by these texts is prohibitive to exploration and experimentation of new
methods. This book includes reproducible examples using real data sets
so you can easily step-through the analysis to explore different
parameters, and easily interchange your own data.

Second, the existing books give little coverage to intuitive and
methodologically more robust methods such as resampling and Bayesian
analyses, favoring traditional parametric methods. This reflects a time
when computational power was scarce, and computations often done by hand
or with a calculator (cf Fletcher and Lock 1991). Modern computers can
now easily handle as resampling and Bayesian methods, and R is unique in
having a mature set of methods for computing common tests in these
frameworks. These methods are also increasingly common in the published
research literature. This book introduces these alternative approaches
to give you more options in your analytical toolkit.

Third, the currently available books are silent on the practical
mechanics of many common data analysis and visualization tasks in
archaeology that are not traditionally considered statistics, such as
displaying and analyzing stratigraphic information from excavation and
spatial data from surveys. Learning a tool-chain for these tasks is
traditionally done on your own, and often at great expense with
proprietary software. This book demonstrates how to write simple
programs that are especially useful for archaeologists. By doing this,
and providing code for other common tasks, this book addresses the need
for instruction in a comprehensive open source tool-chain for these
common tasks in archaeological science. It does this by presenting a
reproducible research environment to show how R and its contributed
packages can be used for start-to-finish research into survey,
excavation, and laboratory data. My hope is that practical and
accessible introduction to reproducible research, such as this book aims
to provide, will improve openness and transparency in archaeology
generally.

This is not a book of detailed discussions of statistical theory, and
you will be referred to other authoritative texts on technical details
of algorithms, etc. This is a book introducing R for practical and
efficient implementations of common tasks in archaeological science, and
serves as a springboard to more advanced R programming. The specific
topics covered in this book are:

\begin{itemize}
\item
  Implementing reproducible research as a practice that is good for
  science generally, and good for your individual productivity
\item
  Working with common archaeological data structures, ingesting them
  into R and manipulating them from messy formats to tidy formats ready
  for analysis
\item
  Visualising spatial data by making maps, doing spatial analysis and
  site classification
\item
  Visualising and analysing stratigraphic data from archaeological
  excavation
\item
  Working with relative and absolute chronologies by doing seriation,
  and calibrating, analysing and visualising radiocarbon dates
\item
  Key methods for working with stone artefacts, faunal remains, pottery,
  glass, and metal artefacts
\item
  Compositional analysis using cluster and principle components analyses
  of multivariate data
\item
  Simplifying collaborative research with version control
\item
  Writing for publication more efficiently with literate programming
\end{itemize}

\section{Why do archaeologists need to use code?}

In recent years, biologists have seen great increases in volumes of
genomic data, due to improvements in sequencing technology. This has
lead them to search for more efficient ways to analyse their data, and
automate their analyses. They found that spreadsheet programs did not
provide enough flexibility to conduct their analyses. As a result, many
have turned to R, Python, and similar programming languages to gain
access to a much wider variety of analytical options. The key detail
here is a shift from the point-and-click interface of a spreadsheet
program, to a command-line interface of a programming language. A
command-line interface refers to interacting with software by sending
instructions to the program as lines of plain text.

While archaeologists don't have the same urgency to manage a data deluge
in the same way biologists have with genomic data, we can benefit from
their efforts to use programming languages for scientific research.
There are two big benefits to using a programming language instead of a
point-and-click interface.

\begin{enumerate}
\def\labelenumi{\arabic{enumi}.}
\item
\end{enumerate}

\section{Why R?}

If you are new to R, you might wonder what makes learning such a quirky
language worthwhile. To me, some of the best features are:

\begin{itemize}
\item
  It's free, open source, and available on every major platform. As a
  result, if you do your analysis in R, anyone can easily replicate it.
\item
  A massive set of packages for statistical modelling, machine learning,
  visualisation, and importing and manipulating data. Whatever model or
  graphic you're trying to do, chances are that someone has already
  tried to do it. At a minimum, you can learn from their efforts.
\item
  Cutting edge tools. Researchers in statistics and machine learning
  will often publish an R package to accompany their articles. This
  means immediate access to the very latest statistical techniques and
  implementations.
\item
  Deep-seated language support for data analysis. This includes features
  likes missing values, data frames, and subsetting.
\item
  A fantastic community. It is easy to get help from experts on the
  R-help mailing list, stackoverflow, or subject-specific mailing lists
  like R-SIG-mixed-models or ggplot2. You can also connect with other R
  learners via twitter, linkedin, and through many local user groups.
\item
  Powerful tools for communicating your results. R packages make it easy
  to produce html or pdf reports, or create interactive websites.
\item
  A strong foundation in functional programming. The ideas of functional
  programming are well suited to solving many of the challenges of data
  analysis. R provides a powerful and flexible toolkit which allows you
  to write concise yet descriptive code.
\item
  An IDE tailored to the needs of interactive data analysis and
  statistical programming.
\end{itemize}

\section{Who should read this book}

This book is written for archaeologists who are keen to expand their
analytical horizons and gain access to new methods and new, more
efficient, ways of organising their research process. No prior knowledge
about R or computer programming is necessary, but a curiosity about
statistics would be an advantage, as well as a readiness to read beyond
this book to make decisions about the suitability of statistical methods
for your specific research questions.

\section{Getting help}

Developing fluency in a language like German or Chinese takes time and
has ups and downs. Learning R is a similar process, and you should
anticipate ups and downs as R becomes an increasingly central part of
your workflow. Fortunately, there are a number of options for seeking
help when you get stuck, or have a question. The simplest case is when
you execute a line of code, only to get a cryptic error message in
reply. Your first action might be to copy the text of the error message
to your clipboard, and paste it into Google. In most cases, the top
search results will be from \href{http://stackoverflow.com/}{Stack
Overflow} or one of the \href{https://www.r-project.org/mail.html}{R
mailing lists}.

Stack Overflow is a free question-and-answer website that is focussed on
computer programming. Participants receive points for asking clear
questions, and for giving useful answers. This adds a competitive
element to participating in the Q\&A, and sometimes when you ask a
well-formed question, several answers will appear very quickly from
participants eager to earn points. The R mailing lists are more
traditional email-based fora, with archives at several places online.
Highly skilled R programmers and R-using scientists from a variety of
disciplines are active on both the R mailing lists and Stack Overflow. I
tend to prefer Stack Overflow because I find it easier to see when a
question has been answered, compared to browsing the email list
archives.

I have found that in the course of writing my question to submit to
Stack Overflow, the process of simplifiying my code into a small,
self-contained example leads me to discover the cause of my problem
(often a misplaced comma or bracket) and I can answer my question before
I need help from others. Similarly, I often find that many of my
questions have already been asked and answered on Stack Overflow - the
challenge is to recognise how similar an existing question is to your
current problem, and thus how useful the answers are to your specific
issue.

If you are sure that your problem is new and unique, and you want to
submit a question, here is some general advice to help you get a useful
answer quickly from the Stack Overflow and R mailing lists communities:

\begin{itemize}
\item
  Spend some time browsing previous posts to understand the commmunity
  norms. Online communities have specific cultural values, and while
  some of these are spelled out in posting guides, many are also
  unwritten, and can only be learnt through mindful reading of previous
  correspondence. If you are familiar with the norms, you are more
  likely to have an efficient and satisfying interaction with the
  community.
\item
  Make sure you have the latest version of R and of the package (or
  packages) you are having problems with. It may be that your problem is
  the result of a recently fixed bug.
\item
  Spend some time creating a reproducible example. This mean a
  simplified version of your problem that another person can copy and
  paste from their broswer into their R console to reproduce the error
  that you see. This can be quite an art-form, but the basic ingredients
  for a good reproducible exaple are: the packages you are using, a
  small data set, code, and a description of your R environment. R comes
  with many example data sets built-in, if you run \texttt{data()} you
  can see a list, and these are very convienent to use for reproducible
  examples. If you want to include your own data, you can use
  \texttt{dput()} to generate R code that another person can copy and
  paste to recreate your dataset. Your code should be easy to read, with
  spaces between operators (+, -, *, etc.) and after commas, and
  conformant to a style guide, such as Hadley Wickham's
  \href{http://adv-r.had.co.nz/Style.html}{guide in Advanced R}. Your
  code should includes lines of commentary, which begin with
  \texttt{\#}, to help others understand your problem. The usual way to
  communicate your R environment (ie. your operating system type and
  version), is to include the output from \texttt{sessionInfo()} into
  your question. Don't worry if some of these terms are unfamiliar at
  the moment, the main thing to know here is that help is available, and
  that the quality of help you receive is proportional to the effort you
  spend seeking it.
\end{itemize}

\section{Prerequisites}

To run the code in this book, you will need to have R installed on your
computer, as well as the RStudio IDE, an application that makes it
easier to use R. Both R and the RStudio IDE are free and easy to
install.

\subsection{R}

To install R, visit
\href{http://cran.r-project.org}{cran.r-project.org}. Then click the
link that matches your operating system. What you do next will depend on
your operating system.

\begin{itemize}
\item
  Mac users should click the most current release. This will be the
  \texttt{.pkg} file at the top of the page. Once the file is
  downloaded, double click it to open an R installer. Follow the
  directions in the installer to install R.
\item
  Windows users should click ``base'' and then download the most current
  version of R, which will be linked at the top of the page.
\item
  Linux users will need to select their distribution and then follow the
  distribution specific instructions to install R.
  \href{https://cran.r-project.org/bin/linux/}{cran.r-project.org}
  includes these instructions along side of the files to download.
\end{itemize}

\subsection{RStudio}

Once you have R installed, it is time to download RStudio. To download
RStudio, visit
\href{http://www.rstudio.com/download}{www.rstudio.com/download}.

Choose the installer for your system. Then click the link to download
the application. Once you have the application, installation is easy.
Once RStudio is installed, open it as you would open any other
application.

\subsection{R Packages}

Some of the most useful parts of R come in \emph{packages}, collections
of functions and code that you can download in addition to base R. We
will use several packages in this book. These include the
\texttt{dplyr}, \texttt{ggplot2}, \texttt{knitr}, \texttt{readr},
\texttt{rmarkdown}, \texttt{stringr}, and \texttt{tidyr} packages, among
others.

There are two general ways to install packages for R. Both require you
to have an internet connection, to start an R session (by opening the
RStudio IDE), and to run a command at the command line.

The most common way to install R packages is to download them from the
package repository at
\href{http://cran.r-project.org}{cran.r-project.org}. To do this run the
command, \texttt{install.packages()}. Give \texttt{install.packages()}
the name or names of the packages you wish to install as a character
vector. R will download the packages from
\href{http://cran.r-project.org}{cran.r-project.org} and install them in
your system library.

You can use this method to download all but one of the packages listed
above. To do so, open R and run the command

\begin{Shaded}
\begin{Highlighting}[]
\KeywordTok{install.packages}\NormalTok{(}\KeywordTok{c}\NormalTok{(}\StringTok{"devtools"}\NormalTok{, }\StringTok{"dplyr"}\NormalTok{, }\StringTok{"ggplot2"}\NormalTok{, }
                   \StringTok{"knitr"}\NormalTok{, }\StringTok{"readr"}\NormalTok{, }\StringTok{"rmarkdown"}\NormalTok{, }
                   \StringTok{"scales"}\NormalTok{, }\StringTok{"stringr"}\NormalTok{, }\StringTok{"tidyr"}\NormalTok{))}
\end{Highlighting}
\end{Shaded}

Some R packages are not stored on
\href{http://cran.r-project.org}{cran.r-project.org}, but are hosted in
online repositories maintained by the package's developer. The most
common place to host these packages is
\href{http://www.github.com}{www.github.com}.

For example, \texttt{aswr} is a collection of functions and data sets
that I have assembled for this book and saved online as a github
repository
(\href{http://github.com/benmarwick/aswr}{github.com/benmarwick/aswr}).

You can install packages stored on github with the
\texttt{install\_github()} function in the \texttt{devtools} package.
(You can install the \texttt{devtools} package itself from
\href{http://cran.r-project.org}{cran.r-project.org} with
\texttt{install.packages()}). To use the function, pass it a
characterstring with the form ```/''.

To install \texttt{aswr}, run the command

\begin{Shaded}
\begin{Highlighting}[]
\NormalTok{devtools::}\KeywordTok{install_github}\NormalTok{(}\StringTok{"benmarwick/aswr"}\NormalTok{)}
\end{Highlighting}
\end{Shaded}

\subsubsection{\texorpdfstring{\texttt{library()}}{library()}}

When R installs a package, it downloads the package to your system
library. This does not automatically load the contents of the package
into your current or future R sessions. To use the functions and data
sets that come in an R package saved in your system library, you must
load the package into your current R session with \texttt{library()}.

For example, to use the functions in the \texttt{tidyr} package, you
would need to first run

\begin{Shaded}
\begin{Highlighting}[]
\KeywordTok{library}\NormalTok{(}\StringTok{"tidyr"}\NormalTok{)}
\end{Highlighting}
\end{Shaded}

You will need to rerun this command each time you open a new R session
in which you wish to use the \texttt{tidyr} package.
